\documentclass[a4paper]{{article}}
\usepackage[a4paper, top=2cm, bottom=2cm, left=1.25cm, right=1.25cm]{{geometry}}
\usepackage[table]{{xcolor}}
\usepackage{{array}}
\usepackage{{tabularray}}
\usepackage{{graphicx}}
\usepackage{{pbox}}

\pagestyle{{empty}}

\setlength{{\parindent}}{{0pt}}

\newcommand{{\colwidth}}{{11pt}}

% Define a new column type for fixed width and center alignment
\newcolumntype{{C}}[1]{{>{{\centering\arraybackslash}}m{{#1}}}}

\begin{{document}}
\begin{{center}}
    \Large Instituto Superior Técnico
    
    Algoritmos e Modelação Computacional

    Relatório de Avaliação
\end{{center}}

O presente documento contém toda a informação relativa à correção automática do exame abaixo mencionado. O exame foi corrigido automaticamente pelo que poderão ter ocorrido erros de leitura. Abaixo o aluno encontrará a chave de correção do seu exame e as respostas que deu a cada pergunta. Uma célula vazia significa que o aluno ou não respondeu à pergunta ou assinalou mais do que uma resposta. Na segunda página deste documento o aluno encontrará a digitalização do seu exame. Apelamos a que o aluno verifique que o seu teste foi corrigido corretamente.

\section*{{Identificação}}
\begin{{tabular}}{{|l|l|}}
    \hline
    \textbf{{Prova}} & {exam_name} \\ \hline
    \textbf{{Data}} & {exam_date}\\ \hline
    \textbf{{Nome}} & {name}\\ \hline
    \textbf{{Número}} & {number}\\ \hline
    \textbf{{Curso}} & {course}\\ \hline
\end{{tabular}}
\section*{{Questões}}

{{\large \textbf{{Versão:}} {version}}}

{tables}

% \vspace*{{0.5cm}}

% \begin{{tabular}}{{|c|C{{\colwidth}}|C{{\colwidth}}|C{{\colwidth}}|C{{\colwidth}}|C{{\colwidth}}|C{{\colwidth}}|C{{\colwidth}}|C{{\colwidth}}|C{{\colwidth}}|C{{\colwidth}}|C{{\colwidth}}|C{{\colwidth}}|C{{\colwidth}}|C{{\colwidth}}|C{{\colwidth}}|C{{\colwidth}}|C{{\colwidth}}|C{{\colwidth}}|C{{\colwidth}}|C{{\colwidth}}|}}
%     \hline
%     \textbf{{Questão}}&\textbf{{1}}&\textbf{{2}}&\textbf{{3}}&\textbf{{4}}&\textbf{{5}}&\textbf{{6}}&\textbf{{7}}&\textbf{{8}}&\textbf{{9}}&\textbf{{10}}&\textbf{{11}}&\textbf{{12}}&\textbf{{13}}&\textbf{{14}}&\textbf{{15}}&\textbf{{16}}&\textbf{{17}}&\textbf{{18}}&\textbf{{19}}&\textbf{{20}}\\
%     \hline
%     \textbf{{Chave}}& A & A & A & A & A & A & A & A & A & A & A & A & A & A & A & A & A & A & A & A\\
%     \hline
%     \textbf{{Resposta}}& A & A & A & A & A & A & A & A & A & A & A & A & A & A & A & A & A & A & A & A\\
%     \hline
% \end{{tabular}}

% \vspace*{{0.5cm}}

% \begin{{tabular}}{{|c|c|c|c|c|c|c|c|c|c|c|c|c|c|c|c|c|c|c|c|c|}}
%     \hline
%     Questão & 1 & 2 & 3 & 4 & 5 & 6 & 7 & 8 & 9 & 10 & 11 & 12 & 13 & 14 & 15 & 16 & 17 & 18 & 19 & 20\\
%     \hline
%     Chave& A & A & A & A & A & A & A & A & A & A & A & A & A & A & A & A & A & A & A & A\\
%     \hline
%     Resposta& A & A & A & \cellcolor{{red!25}}A & A & \cellcolor{{green!25}}A & A & A & A & A & A & A & A & A & A & A & A & A & A & A\\
%     \hline
% \end{{tabular}}

\section*{{Classificação}}
\begin{{center}}
\begin{{tabular}}{{|l|c|}}
    \hline
    \textbf{{Questões respondidas corretamente (\(\mathrm{{QC}}\))}} & {no_correct} \\
    \hline
    \textbf{{Questões respondidas incorretamente (\(\mathrm{{QI}}\))}} & {no_incorrect} \\
    \hline
    \textbf{{Questões não respondidas/resposta inválida (\(\mathrm{{QNR}}\))}} & {no_unanswered} \\
    \hline
    \pbox[c][30pt][c]{{\textwidth}}{{\textbf{{Classificação}}}}& \pbox[c][30pt][c]{{\textwidth}}{{\textbf{{{grade}}}}}\\
    \hline
\end{{tabular}}
\end{{center}}

A clasificação foi obtida usando a fórmula
\[\max(0, (\mathrm{{QC}} - 5) \times 1 )\]

\section*{{Revisão de provas}}
Os erros na correção devem
ser comunicados até dia ??? para o email/google forms ???.

\newpage

\includegraphics[width=\textwidth]{{{path}}}
\end{{document}}